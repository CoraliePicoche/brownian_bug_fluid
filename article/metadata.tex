% DO NOT EDIT - automatically generated from metadata.yaml

\def \codeURL{https://github.com/CoraliePicoche/brownian_bug_fluid/code}
\def \codeDOI{}
\def \codeSWH{}
\def \dataURL{}
\def \dataDOI{}
\def \editorNAME{Pierre de Buyl}
\def \editorORCID{0000-0002-6640-6463}
\def \reviewerINAME{Francesco Turci}
\def \reviewerIORCID{0000-0002-0687-0715}
\def \reviewerIINAME{Rajesh Singh}
\def \reviewerIIORCID{0000-0003-0266-9691}
\def \dateRECEIVED{24 August 2021}
\def \dateACCEPTED{03 May 2022}
\def \datePUBLISHED{13 May 2022}
\def \articleTITLE{[Re] Reproductive pair correlations and the clustering of organisms}
\def \articleTYPE{Replication}
\def \articleDOMAIN{Ecology}
\def \articleBIBLIOGRAPHY{bibliography.bib}
\def \articleYEAR{2022}
\def \reviewURL{}
\def \articleABSTRACT{Abstract is here}
\def \replicationCITE{Young, W. R., Roberts, A. J., & Stuhne, G. (2001). Reproductive pair correlations and the clustering of organisms. Nature, 412(6844), 328-331}
\def \replicationBIB{young_reproductive_2001}
\def \replicationURL{https://www.researchgate.net/profile/William-Young-22/publication/11882165_Reproductive_pair_correlations_and_the_clustering_of_organisms/links/0c96052a9ecabaeec5000000/Reproductive-pair-correlations-and-the-clustering-of-organisms.pdf}
\def \replicationDOI{10.1038/35085561}
\def \contactNAME{Coralie Picoche}
\def \contactEMAIL{coralie.picoche@u-bordeaux.fr}
\def \articleKEYWORDS{phytoplankton, individual-based model, clustering, advection-diffusion-reaction, C++}
\def \journalNAME{ReScience C}
\def \journalVOLUME{8}
\def \journalISSUE{1}
\def \articleNUMBER{3}
\def \articleDOI{}
\def \authorsFULL{Coralie Picoche and Frederic Barraquand}
\def \authorsABBRV{C. Picoche and F. Barraquand}
\def \authorsSHORT{Picoche and Barraquand}
\title{\articleTITLE}
\date{}
\author[1,2,\orcid{0000-0002-0867-2130}]{Coralie Picoche}
\author[1,2,\orcid{0000-0002-4759-0269}]{Frederic Barraquand}
\affil[1]{Institute of Mathematics of Bordeaux, CNRS & University of Bordeaux, Talence, France}
\affil[2]{Integrative and Theoretical Ecology, LabEx COTE, University of Bordeaux, Pessac, France}
